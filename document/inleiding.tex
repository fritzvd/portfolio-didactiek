
\section{Introductie}
Dit is een portfolio van mij, Fritz van Deventer, geschreven voor de cursus: \textit{HBO didactiek} die wordt gegeven vanuit het HAN VDO instituut van de Hogeschool Arnhem en Nijmegen (HAN).
In dit portfolio komen er in het kort een aantal dingen langs die een beeld geven van wat ik geleerd heb.

De cursus wordt gegeven aan de hand van 5 onderwerpen: Didactisch creëren, Begeleiden, Ontwerpen, Beoordelen en \textit{Reflective Practicioner}. Voor elk van de eerste vier onderwerpen is er een hoofdstuk bestaand uit een Persoonlijk Ontwikkelingsplan (POP) en een Zelfreflectie. Het POP eindigt in elk hoofdstuk met een leerdoel geformuleerd uit de observatie van anderen en mijzelf. In iedere Zelfreflectie blik ik terug op de leerdoelen en beschrijf ik hoe ik aan de leerdoelen heb gewerkt, en in hoeverre ik vind dat deze behaald zijn of niet.

Het laatste onderdeel is verbonden met de andere onderwerpen, maar was niet een afzonderlijk gedeelte in de cursus. Dat is het gedeelte \hyperref[sec:RP]{Reflective Practicioner}. Dit is een onderzoek vanuit de onderwijspraktijk, gebruikmakende van kennis van collega's en gesprekken met studenten. Het onderzoek draait om een vraag die ik het liefst beantwoord zie voor mijn dagelijkse beroepspraktijk namelijk: "Hoe verschillend zijn de zienswijze van student en docent ten opzichte van onderwijs en de rol van de docent daarin?".
