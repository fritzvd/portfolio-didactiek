\section{Ontwerpen}

\subsection{Persoonlijk Ontwikkelingsplan}
Aan het begin van deze cursus en het onderdeel van ontwerpen ben ik vrijwel onervaren in het ontwerpen van lesstof. Aan de hand van de lesstof en de reader heb ik het volgende leerdoel geformuleerd:
\begin{quote}
  \textit{Ik zou graag willen leren wat een OWE is en hoe deze opgebouwd is.}
\end{quote}
Dit wil ik graag behalen door een \textit{O}nder\textit{w}ijs\textit{E}enheid (OWE) onder de loep te nemen en te analyseren.

\subsection{Uitpluizen van een OWE}
Voor het ontwerpen van onderwijs ga ik kijken naar de \textit{O}nder\textit{w}ijs\textit{E}enheid (OWE) van het vak \textit{S}tructured \textit{P}rogramming \textit{D}evelopment (SPD). De reden dat ik dit vak heb gekozen is omdat het op het moment het meest relevante vak is voor mijzelf, ik geef het aan 1 klas en het beslaat het meeste van mijn uren voor deze periode. De volledige OWE staat in \hyperref[sec:owespd]{Bijlage F - OWE voor vak SPD}. Verder zal ik hier de OWE even kort toelichten.

De OWE wordt gedefinieerd vanuit een beroepstaak: \textit{Ontwerp, realiseer en test een computerprogramma met gebruikersinteractie aan de hand van een probleemstelling.
}.

De student moet aan de hand van reeks aan functionele eisen een computerprogramma kunnen ontwerpen, de algoritmiek ervoor bedenken, en het ontwerp kunnen implementeren. Bovendien zou de student moeten kunnen beredeneren waarom zo een ontwerp tot zo een progrq
mma heeft geleid en waarom de gestelde eisen tot het ontwerp hebben geleid. De student moet kunnen valideren dat het programma ook daadwerkelijk werkt en voldoet aan de gestelde eisen. Het programma wat geschreven is moet ook gebruik maken van aangeboden API's  en voldoen aan coderinggstandaarden. Niet gebruik maken van overbodige controlestructuren en redundante code. 
Dit kennen, kunnen of doen, valt gedeeltelijk op te hangen aan de competenties en is bijna een complete vervanging voor deze competenties. 

Als we dit kennen, kunnen en doen naast de piramide van Miller zetten, vallen de doelstellingen binnen de knows, knows how en shows how kolommen. De student moet dingen kennen: de beginselen van programmeren, algoritmiek moet bekend zijn. Maar het vak legt ook een duidelijke nadruk op weten hoe het gebruikt moet worden: er moet code geschreven worden en deze moet voldoen aan zekere normen. Ten derde is het beroepsproduct een aantonen van het kunnen van de student.

De toetsing van deze OWE wordt gedaan door middel van twee schriftelijke tentamens en een beroepsproduct. Er zijn open vragen waarin de student een oplossing moet geven voor een gegeven probleemstelling. Het lijkt wat dit betreft op een wiskunde tentamen. Het is zeer moeilijk om dit kunnen te feinzen door dingen uit het hoofd te stampen.

Het beroepsproduct is een opdracht waarin een casus beschreven staat. De student moet vanuit deze opdracht een ontwerp en een implementatie maken. Deze toetsing sluit aan bij de visie.

De visie waarvan uit gewerkt wordt die duidelijk valt te herkennen is het leren door het doen, wat ook past binnen het constructivisme \cite{keursten2006ontwikkeling}.
 
\subsection{Zelfreflectie}
Van te voren had ik nog niet een beeld van hoe een OWE tot stand kwam noch dat er zo een duidelijke link bestond met de beroepscontext per vak. Dat klinkt misschien wat naïef, maar dat ben ik misschien ook wel wat betreft het onderwijs: naïef. De gedachte dat het zinnig is om te leren wordt gedreven vanuit de beroepscontext, deze wordt vertaald naar beroepstaken, deze naar competenties en deze worden uiteindelijk naar een vak vertaald. Dit was een enorme eye-opener voor mij.

% Ontwikkelingsstadia van Luken bekijken 1 theorie kiezen en daaraan ophangen. Of vanuit zelfregulatie
