
\section{Situatieschets Groepsbegeleiding}
\label{sec:groep}
\subsection{Context}
In het eerstejaars project van de Informatica opleiding (I-Project) ben ik procesbegeleider geweest van een groep studenten die een veilingssite moesten maken voor een opdrachtgever (een rol die vervuld werd door een andere docent). De casus en de inhoud van de opdracht waren weggelegd voor de opdrachtgever. Het was belangrijk dat het groepje waar ik begeleiding aan gaf bekend zou worden met de procestool: \textit{Scrum}. Scrum is een manier van werken in teamverband waar er getracht wordt kleine iteraties te maken die elk afgerond worden met een product wat af is. In de praktijk betekent dit veelal het afspreken, plannen en opleveren van de beloofde dingen voor een product in een tijdsperiode van een aantal weken, een zogenoemde \textit{sprint}. Na afloop van elke sprint is er normaliter een \textit{retrospective}, een overleg met elkaar om terug te blikken op hoe het samenwerken gegaan is. Naast dit overleg kwam ik regelmatig langs om even te kijken hoe het ging (en of iedereen wel aanwezig is).

\subsection{Situatie}
In de groepsgesprekken gaf ik meestal een taak mee voor het volgende ontmoetingsmoment. De opdracht die ik gegeven had voor dit specifieke gesprek was om feedback op te schrijven voor elk van de groepsleden, een tip en een top. De opdracht had ik gegeven omdat in sommige van de gesprekken daarvoor ze ook de taak hadden om feedback mee te nemen, maar de meesten deden het uit het hoofd en het leek nogal verzonnen ``on-the-spot``, het had weinig diepgang en iedereen ging elkaar wat napraten. Ik had nadrukkelijk genoemd dat iedereen echt wat moest opschrijven, anders zou ik zelf weglopen bij het gesprek, omdat het dan ook geen zin had om feedback met elkaar te bespreken.

De volgende afspraak had iedereen wat opgeschreven en meegenomen. Toen we de ronde deden was een van de leden van de groep (we geven hem even de fictieve naam: Hendrikus) nog steeds terughoudend met het geven van de tip. Hendrikus noemde steeds dat de anderen het zo goed doen en dat hij er niet echt iets had op aan te merken. Op dat moment leek het mij verstandig om een zijstapje te doen en te communiceren over het communiceren met de groep. Ik vroeg ze of ze het prima vonden als Hendrikus iets over hun zou zeggen waar ze aan moesten werken of wat niet louter positief was. De groep antwoordde dat ze daar echt niet bang voor waren, dit verwoordde ze ook naar Hendrikus, dat hij zich daar niet zorgen over hoefde te maken. Ik vroeg ook Hendrikus of hij misschien wist waarom hij het niet zo goed over zijn lippen kreeg. Had hij dan geen feedback? Of vond hij het spannend? 

Op dat moment leek het alsof er even een schilletje om Hendrikus wegviel. Hij noemde dat hij het inderdaad ingewikkeld vond om iets negatiefs (ook al was het opbouwend bedoeld) te zeggen omdat hij in zijn schoolloopbaan redelijk veel gepest werd. Hij was bang dat hij het vertrouwen of zijn goede band met de anderen zou schaden door iets te noemen wat niet positief was. Nadat hij dit noemde reageerde de groep accepterend en herhaalde nogmaals dat hij zich er niet zorgen om hoefde te maken. Waarna Hendrikus ook zijn tips durfde te delen.

Wat ik merkte in dit gesprek is dat het echt wel loont om soms even een zijstap te doen en te durven door te vragen. Wat er nog wel bij gezegd moet worden, is dat in deze groep een andere jongen de groep heeft moeten verlaten die kampte met depressie in combinatie met een Autisme Spectrum stoornis.