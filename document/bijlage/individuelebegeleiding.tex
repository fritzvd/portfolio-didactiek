
\section{Situatieschets Individuele begeleiding}
\label{sec:individu}
Één van de studenten die ik begeleid voor zijn afstuderen had een projectplan opgestuurd wat nogal onder de maat was. Erg algemene risico's niet echt specifieke doelen om te behalen en onderzoeksvraag waar niet echt mee gewerkt kon worden. Naar aanleiding van zijn projectplan en om even kennis te maken met de student, het afstudeerbedrijf en zijn begeleiders had ik een afspraak gemaakt met hem.

De afspraak begon helaas een beetje scheef omdat ik vergeten was de feedback die ik voor hem had verzameld een aantal dagen voor de afspraak naar hem te sturen. De student was een beetje verrast door mijn terugkoppeling op zijn document en de onderdelen die ik nog te onduidelijk vond. Het overviel hem even op dat moment, dat was totaal niet mijn bedoeling. 

Daarna heb ik geprobeerd door wat vragen te stellen duidelijker te krijgen wat de situatie was en wat de doelstelling van de opdracht was. Op een gegeven moment heb ik het gesprek afgerond en voorgesteld om nog even met de student individueel te zitten en samen te kijken naar de onderzoeksvraag en het werk wat er nog moet gebeuren.

De gesprekstechnieken die besproken zijn in de lessen en de oefening met de acteur waren mij allemaal even ontschoten en het kostte mij redelijk wat tijd om die weer op orde te krijgen. De taak als begeleider is naar mijn idee om de student zelf aan het denken te zetten. Op het moment dat de situatie dan geheel anders loopt, doordat de student de feedback pas voor het eerst hoorde toen ik bij hem was, dan is het toch een stuk lastiger om het ook daadwerkelijk bij de student te laten en niet zelf heel hard te gaan werken. 

Gezamenlijk hadden we afgesproken om elkaar nog telefonisch te spreken over het projectplan een week later. Tijdens dat gesprek was het een stuk makkelijker om de "goede" vragen te stellen en de student aan het werk te zetten. Hier blijkt voor mij duidelijk uit dat het belangrijk is om die gesprekstechnieken veelvuldiger te oefenen en te gebruiken zodat het een tweede natuur wordt, ook voor de onvoorziene situaties.
