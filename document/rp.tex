
\section{Reflective Practicioner}
\label{sec:RP}
\subsection{Introductie}
Voor het onderzoeksgedeelte heb ik mij gericht op wat studenten en docenten als meest waardevol beschouwen in de les. Waar hebben ze het meeste van geleerd, ten opzichte van waarvan de docenten denken wat het meest waardevol is om te doen. De onderzoeksvraag is:
\begin{quote}
  "Hoe verschillend zijn de zienswijze van student en docent ten opzichte van onderwijs en de rol van de docent daarin?"
\end{quote}
De onderzoeksvraag is ontstaan door een interesse die gewekt is na het lezen van een aantal meta-analyses in het onderwijs-onderzoeksveld \cite{hattie2008visible, schneiderVariables}. Beiden analyses die noemen eigenlijk dat de meeste onderwijsmethoden een positief effect hebben op leren. Maar er zijn sommige die een veel grotere impact hebben dan anderen, zoals: open vragen t.o.v. gesloten vragen stellen, sleutelwoorden op slides zetten t.o.v. volzinnen \cite{schneiderVariables}. Omdat student en docent het onderwijs wat ze volgen of geven heel anders ervaren, is het interessant om te kijken of hoe groot de verschillen zijn in wat beiden partijen waarnemen als methoden die het leereffect verhogen. Niet puur uit een wetenschappelijke interesse, maar ook om dat weer te gebruiken in de praktijk.

\subsection{Methodiek}
Aan de hand van een vragenlijst die beschreven staan in \hyperref[sec:vragenlijstRP]{Bijlage E - Vragenlijsten} heb ik geprobeerd te kijken wat de verschillen zijn tussen studenten en docenten door dezelfde onderwerpen vragen in een vraag geformuleerd voor de student en één geformuleerd voor de docent. Het is niet een wetenschappelijk opgestelde vragenlijst, noch zal de analyse een wetenschappelijke aard hebben.

De vragen zijn op een schaal van nooit-altijd beantwoord. Bij sommige van de vragen heb ik alle antwoorden die positief/negatief zijn bij elkaar gegooid het onderscheid daartussen iets scherper te krijgen.

\subsection{Bevindingen}
Ten eerste is mij duidelijk dat de vragen die ik heb gesteld niet altijd duidelijk waren, en in sommige opzichten onvolledig. Dus ik zal het even moeten doen met de vragen die gesteld zijn. 

\subsubsection{Colleges}
Eigenlijk zijn zowel studenten (92.6\%) en docenten (100\%) het er mee eens dat de colleges meer dan soms nuttig zijn. Opvallend is dan wel dat bijna de helft van de studenten (43.1\%) vindt dat zij overwegend niet veel hoeven in te halen als ze een college gemist hebben, terwijl de docenten (90\%) dat overwegend wel vinden.

\subsubsection{``Leuk''}
Alle docenten proberen over het algemeen het vak leuk te maken (100\%), de meeste studenten (87\%) vinden dat een docent daar ook in bepalend kan zijn voor een vak. En daarin hebben juist docenten (100\%) vaker het idee dat dat effect heeft dan studenten (92\%).

\subsubsection{Tempo}
Het meest verdeeld lijken de studenten en docenten over het tempo van de les. Als een docent te snel gaat dan haakt 51.3\% soms niet/wel, 18.8\% vaak en 8.8\% altijd af. Terwijl als een docent te langzaam gaat haakt 50\% soms niet/wel, 18.8\% vaak en 11.3\% altijd af.

Het lastige hierbij is dat van de docenten 61\% aangeeft niet altijd door te hebben dat ze te snel gaan en daarmee de studenten af haken. En ongeveer evenveel (60\%) docenten aangeven niet altijd gas terug te nemen als ze te snel gaan.


\subsection{Conclusie}
\cite{lalley2007learning}