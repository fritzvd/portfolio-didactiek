
\section{Activerende Didactiek}
\subsection{Persoonlijk Ontwikkelingsplan}
\label{sec:didactiek}

Aan de hand van mijn worst-case scenario en een SWOT-analyse heb ik een leerdoel ontwikkeld. De SWOT-analyse is uitgewerkt in \hyperref[tab:SWOTDidactiek]{Tabel 1}.

De SWOT-analyse is opgebouwd uit feedback die ik heb gekregen van anderen, uit dingendie naar voren zijn gekomen in de lesbezoeken, de lesdemo en een aha-erlebnis die ik uit mijn worst-case scenario heb gehaald.


\begin{table}[h]
  \begin{center}
  \hspace*{-2cm}
  \begin{tabular}{ l | p{6cm} | p{6cm} }
     \hline
      & behulpzaam & schadelijk \\ \hline
     intern & 
       \begin{itemize}[noitemsep, leftmargin=*]
         \item{Gebruik van voorbeelden en metaforen} 
         \item{Zekere indruk.}
         \item{Vragen positief bekrachtigen met complimenten.}
         \item{Laagdrempelig}
         \item{Heldere duidelijke stem. Prettig aanwezig}
       \end{itemize}
     & 
       \begin{itemize}[noitemsep, leftmargin=*]
         \item{Te weinig tijd geven voor een antwoord}
         \item{De belangrijke dingen verbaliseer ik te weinig.}
         \item{Niet rekening houden met andere leerstijlen.}
         \item{Te weinig context bieden, bij een sprong in het diepe.}
         \item{Niet genoeg tijd nemen voor lesvoorbereiding.}
       \end{itemize}
         \\ \hline
     extern & 
       \begin{itemize}[noitemsep, leftmargin=*]
         \item{Over het algemeen is er genoeg tijd om lessen voor te bereiden, waarmee er ook een diversiteit in werkvormen valt te onderzoeken.
         \item De lessen zijn allemaal lang (3 uur), hierin is genoeg tijd om context te geven rond een thema, een werkvorm te doen en daarop te reflecteren tijdens de les.}
       \end{itemize}
       & 
       \begin{itemize}[noitemsep, leftmargin=*]
         \item{Iedereen zit in de les met een laptop voor hun neus, de kans tot afleiding is en de kans dat dingen mij ontgaan die de klas bezighouden zijn groot.}
         \item{De afstand tussen de student en mij in jaren groeit.}
       \end{itemize}
       \\
     \hline
   \end{tabular}
   \caption{SWOT analyse. Met de klok mee: S, W, T, O.}
   \label{tab:SWOTDidactiek}
  \end{center}
 \end{table}

\subsubsection{Worst-case scenario}
De worst-case is voorgevallen op een middelbare school waar ik lesgaf aan een VWO 5 klas. Richting het einde van het jaar was er een hoofdstuk wat inzoomde op de geschiedenis van Informatica. Daarmee werd het minder een oefening van nieuwe vaardigheden zoals dat met andere onderwerpen zoals programmeren en databases wel het geval was. Maar vooral een kwestie van begrijpend lezen en het onthouden van wat feiten.

De stof zelf was niet heel inspirerend, dat was misschien ook wel duidelijk af te lezen aan mijn houding. Daarenboven had ik mijn lessen niet altijd goed voorbereid met een onderbreking of iets leuks. Om de powerpoints en de stof op te leuken, had ik hier en daar een filmpje ter ondersteuning gezocht. Wat grapjes en wat anecdotes om de slides te ondersteunen. Helaas was de desbetreffende les een tranendal. Gedurende een 20 minuten durende presentatie, werd het steeds rumoeriger en begon ik steeds meer tegen de ruggen dan tegen de gezichten van de leerlingen aan te kijken. Ik maakte even een zijstapje. Ik parafraseer:
\begin{quote}
  \textit{"Wat gebeurt er precies? Ik snap dat dit misschien niet het meest interessante onderwerp is en dat het bijna vakantie is. We doen dit samen, als jullie meedoen gaat het een stuk sneller."}
\end{quote}

Vervolgens werd er 1 minuut stilte geveinsd waarna de houding van daarvoor weer een schepje er bovenop kreeg. Op dat moment werd ik zelf wat sarcastisch en begon snel door de slides heen te klikken waarbij ik dingen riep als: \textit{"Kijk, een plaatje. leuk hé?"}. Dit schouwspel verstomde de klas tot een allesomvattende stilte, waarbij ik het eindigde met: \textit{"Ik haal even koffie."}

Daarna heb ik mijzelf vermand, het was een blokuur en de bel van het eerste van de twee uur was nog niet gegaan. Toen heb ik wat interessante dingetjes opgezocht een kahoot gemaakt en wat rondgelopen om leerlingen te praten over hun buitenlandexcursie.

Wat mij hierin opviel is dat ik het heel vervelend vond dat ik totaal de aandacht en het contact met leerlingen was verloren. Wat eigenlijk het hele jaar goed was gegaan als beginnende docent liep nu finaal in de soep. Dit verwijt ik grotendeels mijzelf. Door het onderwijs inmiddels iets bewuster te begrijpen vanuit onderwijsfuncties \cite[p.45-62]{kallenberg2014leren} zie ik dat ik qua cognitie en affect redelijk vanuit instinct heb kunnen handelen. Maar het ontbreekt mij nog aan regulatieve maatregelen die iets meer proces en structuur aan de lessen geven. Om die verschillende onderwijsfuncties bewuster in te vullen wil ik gebruik maken van het lesplan, dat de verschillende fasen van een les onderverdeeld en taken definieert die passen binnen de verschillende onderwijsfuncties en tracht te voorkomen dat belangrijke didactische stappen worden overgeslagen \cite[p.163-169]{bijkerk2015activerend}.
%Deze stof had ik zelf niet iets leuks van gemaakt. Waar dat misschien wel mogelijk was. Hierin zat mijn grootste falen in de voorbereiding. Ik bereid niet altijd mijn lessen goed voor, waardoor ik soms aangewezen ben op mijn improvisatievaardigheden. Door deze worst-case te ondergaan ben ik er van overtuigd dat ik zelfs de lessen die mij makkelijk af zouden moeten gaan gedegen moet voorbereiden.

\subsubsection{Leerdoel}
Het leerdoel wat ik uit de SWOT-analyse en de worst-case scenario heb gedistilleerd is als volgt:
\begin{quote}
  De lessen en lesstof die mij minder interesseren ga ik meer aandacht geven dan de lesstof die \textit{"leuk"} is.
\end{quote}

Dit leerdoel wil ik gaan behalen door voor de eerste 4 lessen van het nieuwe blok (P1 2017/2018) een planning te maken aan de hand van het lesplan. Hiermee dwing ik mijzelf om wat ruimer van te voren de tijd te nemen om over mijn les na te denken. In de hoofdfase van de les wil ik dan graag gebruik maken van verschillende activerende werkvormen die om het leren op gang te brengen het leren combineren met het opdoen van concrete ervaringen zoals in de leercyclus van Kolb \cite[p.143-145]{kallenberg2014leren}. 

\subsection{Zelfreflectie}
\label{sec:Zelfreflectie}
Hier zal ik terugblikken op het leerdoel welke ik heb geformuleerd en hoe ik dat heb getracht te behalen. De reflecties zowel als de feedback formulieren van de lesdemo en het lesbezoek van de docent staan in Bijlage \hyperref[sec:lesdemo]{A}, \hyperref[sec:lesbezoek]{B} en \hyperref[sec:lesbezoekcollega]{C}..

De afgelopen 4 lessen heb ik nu met behulp van het lesplan vormgegeven, wat een hele uitdaging bleek, want structureren gaat bij mij niet zo vanzelf. Ik ben vrij goed in improviseren en vertrouw er dan ook op dat ik als een onderdeel minder goed heb voorbereid, dat ik er dan ook wel uitkom zonder die voorbereiding. Dat is een sterke kant van mij, maar niet een kant die ik wil misbruiken in dienst van mijn eigen luiheid. Het voordeel is dat wanneer zich een onvoorziene situatie voordoet dat ik die goed kan anticiperen en daarop handelen. Het nadeel is echter dat ik niet altijd duidelijk heb wat ik de studenten precies wil meegeven en hoe ik dat ga doen.
Als ik dat niet goed voorbereid dan ben ik zo bezig met improviseren dat ik dan te weinig ondertitel van wat we doen en waarom we dat doen in de les. Wat precies dat regulatieve gedeelte in het gevaar brengt.

Toen ik bezig ging met het lesplan tijdens de cursus zag ik daar wel wat het voordeel er van zou kunnen zijn. Maar in de dagelijkse praktijk voelt dat omslachtig. Eerst een lesplan en van daaruit een les voorbereiden.

Desalniettemin ben ik er mee aan de slag gegaan voor mijn leerdoel. Ik ben elke voorbereiding gestart met het rudimentair opschrijven van het doel, de orde van leren die het doel tracht te behalen volgens de taxonomie van Bloom en de methoden aan de hand van de lesinhoud. Tijdens de voorbereiding begon ik dan ook wat slides te maken om de les wat te omlijsten. Dat alleen al gaf een kapstop om de les in te delen. De hulpwoorden: \textit{duo's, plenair, carousel} hielpen mij om in ieder geval iets anders te bedenken dan een individueel of plenair deel. Hiermee heb ik kunnen oefenen met wat verschillende activerende werkvormen. Zo heb ik de studenten aan elkaar laten verwoorden welke nieuwe concepten ze de voorgaande lessen hebben geleerd en hoe ze die kunnen gebruiken. Of kleine opdrachten gegeven die analoog zijn aan de resultaten die worden verwacht bij het uiteindelijke beroepsproduct van het vak. Zo hebben ze tijdens de les een programma moeten maken dat een dambord met damstukken tekent. Dit sluit ook aan bij de handelingspsychologie leertheorie \cite[p.417]{kallenberg2014leren}. Die stelt dat het leren bestaat uit een materiële (denk aan het dambord), een perceptieve (zoek het dambord op op internet), verbale (beschrijf aan elkaar waar een dambord uit bestaat) en een mentale handeling (aan welke regels voldoet een dambord). 

In vrijwel elke les verwerk ik een kleine competitieve quiz met Kahoot, waarbij ik de stof van vorige week ophaal door middel van vragen te stellen met code-voorbeelden. Eenmaal in de week zijn de studenten zelf aan de beurt om vragen te maken. Ten eerste vinden ze het heel leuk dat hun eigen vraag langskomt, dus zijn ze volledig geëngageerd en ze oefenen nogmaals met het formuleren van hun kennis, wat ook als 1 van de top-strategieeen van het leren wordt genoemd in: \textit{"What Works, What Doesn't?"} \cite{dunlosky2013}.

Het gebruiken van activerende werkvormen wordt steeds gemakkelijker om vanzelf op te pakken. De \textit{grabbelton van ervaring} wordt steeds rijkelijker gevuld en is in dat opzicht weer complementair aan mijn improvisatievaardigheden.

Door het lesplan te gebruiken, denk ik van te voren na over wat ik welk leerresultaat ik beoog. Ik denk er over na waar ik mee wil beginnen en wat ik wil behandelen. Al met al ben ik heel blij met mijn gekozen leerdoel en hoe het is uitgepakt. Om heel eerlijk te zijn verwacht ik niet elke les met behulp van het lesplan te gaan voorbereiden, maar het helpt mij in ieder geval aan het begin van een nieuw vak om wat richting te geven in hoe ik mijn les indeel en wat ik erbij pak.
Waar ik nog steeds te kort in schiet desondanks het gebruik van het leerplan is de afsluiting. Ik heb gemerkt dat ik niet altijd goed terugkoppel wat we hebben behandeld, ik blik niet terug op de leerdoelen en kom daar verder weinig op terug. Dat zou ik graag in de toekomst willen veranderen door de komende 3 lessen juist op dat laatste stuk van de les te focussen in de voorbereiding. Maar de evaluatie daarvan is helaas niet bestemd voor dit document.

\subsection{Ontvangst}
Aan het eind van elk vak wordt er een docentenevaluatie afgenomen. In deze evaluatie komt naar voren dat de studenten zeer tevreden zijn geweest met mijn lessen, ik denk dat ook ten dele komt door de extra voorbereiding die ik erin heb gestopt. Hieronder zijn de resultaten te vinden van die evaluatie.
\begin{table}[h]
  \begin{center}
  \hspace*{-2cm}
  \begin{tabular}{ l | p{10cm} | p{2cm} }
    
    1. & Deze docent is vakinhoudelijk deskundig & 4.86	 \\
    2. & Deze docent is didactisch vaardig	& 4.9 \\
    3. & Deze docent heeft voldoende kennis van de beroepspraktijk	 & 4.43 \\
    4. & De bijeenkomsten van deze docent ervaar ik als zinvol & 4.85 \\ 
    5. & Deze docent toont zich in studenten geïnteresseerd & 4.8 \\
    6. & Ik kan zonder problemen met mijn vragen bij deze docent terecht & 4.8 \\
    7. & Deze docent geeft bruikbare feedback op mijn leerprestaties & 4.5 \\
    8. & Deze docent inspireert / motiveert mij om een goede beroepsbeoefenaar te worden	& 4.45 \\
  \end{tabular}
  \caption{Docentenevaluatie vak SPD.}
  \label{tab:docentenevaluatie}
  \end{center}
\end{table}

\begin{itemize}
  \item Docent is zeer gericht op de lesstof die voor de huidige week staat gepland. Uitlopers krijgen minder feedback op hun opdrachten.
  \item Erg jong, erg slim, leuk om naar te luisteren, legt duidelijk de stof uit en had de eerste dag van les geven al alle namen van onze klas in zijn hoofd, heel erg betrokken en aardig!
  \item Geen opmerkingen, Fritz is een gemotiveerde docent en je kunt bij Fritz veel leren.
  \item Goede leraar, legt goed en duidelijk uit.
  \item Frits is mijn idee van een perfecte docent
  \item Goede docent! Legt duidelijk uit en blijft herhalen en vragen of het duidelijk is
\end{itemize}

% Om tot een goed leerdoel te komen, heb ik gebruik gemaakt van een SWOT matrix (zie Tabel~\ref{tab:SWOTDidactiek}), waarin ik Strengths (S), Weaknesses (W), Opportunities(O) en Threats(T) heb ingedeeld op behulpzaam (SO) en schadelijk (WT) en intern (SW) en extern (OT). Met intern bedoel ik de dingen in mij zelf, waar ik zelf invloed op kan uitoefenen en met extern de dingen die vanuit anderen komen en waar ik minder invloed op kan uitoefenen. 



