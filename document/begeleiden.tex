\section{Begeleiden}
\subsection{Persoonlijk Ontwikkelingsplan}

Bij het begeleiden van stages en afstuderen merk ik dat ik veel op zoek ben naar een verdieping van het werk. Bijvoorbeeld: De student start een onderzoek van 6 maanden maar heeft een onderzoeksvraag die in een week opgelost kan zijn. Niet iedereen doet dit automatisch goed. Niet alle studenten ontwikkelen hetzelfde. Ik wil graag daarin gebruik maken van de ontwikkelingsstadia, omdat die een redelijk goede indicator blijken voor het inschatten van plannings- en andere zelfregulatievaardigheden \cite{luken2008mogelijkheid}. Maar ik merk dat ik schipper tussen de verantwoordelijkheid bij de student laten en soms te veel invullen of doen voor de student. Dan wil ik de gesprekstechnieken, zoals het GROW-model toepassen om te kijken of er een betere onderzoeksvraag boven tafel kan komen.

Mijn leerdoel wordt dus:
In het komende contact voor een afstudeer- en/of stagebegeleiding wil ik door middel van gebruik van het GROW-model komen tot een verdieping voor de student, vooral als de student zelf niet in staat lijkt tot zelfregulatie. Een verdieping kan dan zijn: een realistischere planning, een betere onderzoeksvraag of iets dergelijks.

\subsection{Zelfevaluatie}
In de bijlagen \hyperref[sec:groep]{C} en \hyperref[sec:individu]{D}. worden twee situaties aangedragen. Allereerst zou ik even de individuele situatie van de afstudeerder even willen uitlichten aan de hand van het estafetteloop-model \cite{methorstkaders}. Daarna behandel ik de groepssituatie aan de verschillende niveaus van reflectie van Korthagen \cite{korthagen2002niveaus}.

\subsubsection{Individuele situatie}
Het estafetteloop-model onderscheidt de fases van een opdracht in uitvoering bij een bedrijf met de metafoor van een estafetteloop. Waarin de opdracht het stokje voorstelt dat doorgegeven moet worden van bedrijf naar student weer terug naar het bedrijf. Als ik de opdracht van de student bekijk en de manier waarop hij in het begin bezig is geweest met het projectplan krijg ik het idee dat het doel en de eisen vanuit mij en vanuit de opdrachtgever wellicht niet duidelijk genoeg zijn geweest. 
Alhoewel de opdracht voldoet aan de kerneis van een praktijkopdracht binnen het idee van \textit{Slow Advice}: niet urgent, wel belangrijk \cite{methorstslow} is het niet duidelijk geworden voor de student welke onderzoeksvraag nou beantwoord moet worden. Als ik achteraf het gesprek bekijk in het licht van het estafetteloop-model zie ik dat ik een aantal dingen anders had moeten of kunnen doen.

Ten eerste had ik duidelijk moeten maken waar afstuderen aan moet voldoen. Zodat het doel van het document en zijn werk duidelijk is.
Ten tweede had ik de begeleiders aan het woord moeten laten door middel van vragen te stellen over het te worden opgeleverde product, bijvoorbeeld: \textit{Wat hebben jullie nodig om het een succes te maken?}. 

\subsubsection{Groepssituatie}
\emph{Fase 1: Handelen}

Het doel van de begeleiding van de projectgroep was enerzijds om de groep inzicht te geven in hoe er gewerkt wordt en anderzijds om het geven van feedback een plek te geven. Dit groepsoverleg was speciaal daarvoor gepland om dat bij de voorgaande groepsmomenten de feedback weinig diepgang had. Door iedereen wat op te laten schrijven en daarover door te vragen wilde ik voor iedereen een succes-ervaring creeëren in het geven van feedback.

\noindent \emph{Fase 2: Terugblikken op het handelen}
Een van de studenten probeerde onder de confrontatie uit te komen door alleen maar positieve dingen te noemen, wellicht voelde hij zich kwetsbaar of niet veilig, of dacht hij dat zijn ligging in de groep in het geding kwam door negatieve feedback te delen.
Door te intervenieren op het groepsgesprek heb ik getracht inzichtelijk maken dat deze persoon ook medeverantwoordelijk is voor het groepsproces. De condities voor het groepsproces kunnen zo gewaarborgd worden \cite{baert2009interventies}.

\noindent \emph{Fase 3: Bewust worden van essentiële aspecten}

Als ik terugkijk is het misschien niet duidelijk geweest wat de condities zijn voor een succesvol groepsproces. Als ik kijk naar de theorie is dat precies de rol die ik op mij had moeten nemen \cite{baert2009interventies}. Het ``contract'' was in die zin niet helder. Dat ga ik volgende keer anders doen door wat afspraken op te stellen van te voren.

\noindent \emph{Fase 4: Formuleren van alternatieven}

Natuurlijk ga ik niet een letterlijk contract opstellen, maar ik ga het wel duidelijker maken voor de leden voor de groep dat het gesprek er niet zozeer voor mij is maar voor hun eigen ontwikkeling. Bovendien denk ik dat het zou helpen om voor een aantal van de gesprekken een andere vorm te kiezen. Aan de hand van een metafoor of iets dergelijks (bijv. de groep is het schip, wat is het anker, wie is de stuurman etc.), ook wel de zijdeurinterventie genoemd \cite{remmerswaal2015zijdeurinterventies}.
