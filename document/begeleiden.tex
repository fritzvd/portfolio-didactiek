\section{Begeleiden}
\subsection{Persoonlijk Ontwikkelingsplan}

Bij het begeleiden van stages en afstuderen merk ik dat ik veel op zoek ben naar \textit{een verdieping} van het werk, die nog ontbreekt. Bijvoorbeeld: De student start een onderzoek van 6 maanden maar heeft een onderzoeksvraag die in een week opgelost kan zijn. Niet iedereen doet dit automatisch goed. Dat is ook niet vreemd, niet alle studenten ontwikkelen zich hetzelfde. Ik wil graag daarin gebruik maken van de psychologische ontwikkelingsstadia, omdat die een redelijk goede indicator blijken voor het inschatten van plannings- en andere zelfregulatievaardigheden \cite{luken2008mogelijkheid}. Maar ik merk dat ik schipper tussen de verantwoordelijkheid bij de student laten en soms te veel invullen of doen voor de student. Dan wil ik de gesprekstechnieken, zoals het GROW-model toepassen om te kijken of er een betere onderzoeksvraag boven tafel kan komen.

Mijn leerdoel wordt dus:
In het komende contact voor een afstudeer- en/of stagebegeleiding wil ik door middel van gebruik van het GROW-model komen tot een verdieping voor de student, vooral als de student zelf niet in staat lijkt tot zelfregulatie. Een verdieping kan dan zijn: een realistischere planning, een betere onderzoeksvraag of iets dergelijks.

Hieronder beschrijf ik twee situaties, één van een individuele begeleiding en één van een groepsbegeleiding. Hierna beschrijf ik hoe het contact vond gaan en probeer ik in te zoomen op hoe ik mijn leerdoel heb getracht te proberen te halen.

\subsection{Situatieschets Individuele begeleiding}
\label{sec:individu}
Één van de studenten die ik begeleid voor zijn afstuderen had een projectplan opgestuurd wat nogal onder de maat was. Erg algemene risico's niet echt specifieke doelen om te behalen en onderzoeksvraag waar niet echt mee gewerkt kon worden. Naar aanleiding van zijn projectplan en om even kennis te maken met de student, het afstudeerbedrijf en zijn begeleiders had ik een afspraak gemaakt met hem.

De afspraak begon helaas een beetje scheef omdat ik vergeten was de feedback die ik voor hem had verzameld een aantal dagen voor de afspraak naar hem te sturen. De student was een beetje verrast door mijn terugkoppeling op zijn document en de onderdelen die ik nog te onduidelijk vond. Het overviel hem even op dat moment, dat was totaal niet mijn bedoeling. 

Daarna heb ik geprobeerd door wat vragen te stellen duidelijker te krijgen wat de situatie was en wat de doelstelling van de opdracht was. Op een gegeven moment heb ik het gesprek afgerond en voorgesteld om nog even met de student individueel te zitten en samen te kijken naar de onderzoeksvraag en het werk wat er nog moet gebeuren.

De gesprekstechnieken die besproken zijn in de lessen en de oefening met de acteur waren mij allemaal even ontschoten en het kostte mij redelijk wat tijd om die weer op orde te krijgen. De taak als begeleider is naar mijn idee om de student zelf aan het denken te zetten. Op het moment dat de situatie dan geheel anders loopt, doordat de student de feedback pas voor het eerst hoorde toen ik bij hem was, dan is het toch een stuk lastiger om het ook daadwerkelijk bij de student te laten en niet zelf heel hard te gaan werken. 

Gezamenlijk hadden we afgesproken om elkaar nog telefonisch te spreken over het projectplan een week later. Tijdens dat gesprek was het een stuk makkelijker om de "goede" vragen te stellen en de student aan het werk te zetten. Hier blijkt voor mij duidelijk uit dat het belangrijk is om die gesprekstechnieken veelvuldiger te oefenen en te gebruiken zodat het een tweede natuur wordt, ook voor de onvoorziene situaties.


\subsection{Situatieschets Groepsbegeleiding}
\label{sec:groep}
\subsubsection{Context}
In het eerstejaars project van de Informatica opleiding (I-Project) ben ik procesbegeleider geweest van een groep studenten die een veilingssite moesten maken voor een opdrachtgever (een rol die vervuld werd door een andere docent). De casus en de inhoud van de opdracht waren weggelegd voor de opdrachtgever. Het was belangrijk dat het groepje waar ik begeleiding aan gaf bekend zou worden met de procestool: \textit{Scrum}. Scrum is een manier van werken in teamverband waar er getracht wordt kleine iteraties te maken die elk afgerond worden met een product wat af is. In de praktijk betekent dit veelal het afspreken, plannen en opleveren van de beloofde dingen voor een product in een tijdsperiode van een aantal weken, een zogenoemde \textit{sprint}. Na afloop van elke sprint is er normaliter een \textit{retrospective}, een overleg met elkaar om terug te blikken op hoe het samenwerken gegaan is. Naast dit overleg kwam ik regelmatig langs om even te kijken hoe het ging (en of iedereen wel aanwezig is).

\subsubsection{Situatie}
In de groepsgesprekken gaf ik meestal een taak mee voor het volgende ontmoetingsmoment. De opdracht die ik gegeven had voor dit specifieke gesprek was om feedback op te schrijven voor elk van de groepsleden, een tip en een top. De opdracht had ik gegeven omdat in sommige van de gesprekken daarvoor ze ook de taak hadden om feedback mee te nemen, maar de meesten deden het uit het hoofd en het leek nogal verzonnen ``on-the-spot'', het had weinig diepgang en iedereen ging elkaar wat napraten. Ik had nadrukkelijk genoemd dat iedereen echt wat moest opschrijven, anders zou ik zelf weglopen bij het gesprek, omdat het dan ook geen zin had om feedback met elkaar te bespreken.

De volgende afspraak had iedereen wat opgeschreven en meegenomen. Toen we de ronde deden was een van de leden van de groep (we geven hem even de fictieve naam: Hendrikus) nog steeds terughoudend met het geven van de tip. Hendrikus noemde steeds dat de anderen het zo goed doen en dat hij er niet echt iets had op aan te merken. Op dat moment leek het mij verstandig om een zijstapje te doen en te communiceren over het communiceren met de groep. Ik vroeg ze of ze het prima vonden als Hendrikus iets over hun zou zeggen waar ze aan moesten werken of wat niet louter positief was. De groep antwoordde dat ze daar echt niet bang voor waren, dit verwoordde ze ook naar Hendrikus, dat hij zich daar niet zorgen over hoefde te maken. Ik vroeg ook Hendrikus of hij misschien wist waarom hij het niet zo goed over zijn lippen kreeg. Had hij dan geen feedback? Of vond hij het spannend? 

Op dat moment leek het alsof er even een schilletje om Hendrikus wegviel. Hij noemde dat hij het inderdaad ingewikkeld vond om iets negatiefs (ook al was het opbouwend bedoeld) te zeggen omdat hij in zijn schoolloopbaan redelijk veel gepest werd. Hij was bang dat hij het vertrouwen of zijn goede band met de anderen zou schaden door iets te noemen wat niet positief was. Nadat hij dit noemde reageerde de groep accepterend en herhaalde nogmaals dat hij zich er niet zorgen om hoefde te maken. Waarna Hendrikus ook zijn tips durfde te delen.

Wat ik merkte in dit gesprek is dat het echt wel loont om soms even een zijstap te doen en te durven door te vragen. Wat er nog wel bij gezegd moet worden, is dat in deze groep een andere jongen de groep heeft moeten verlaten die kampte met depressie in combinatie met een Autisme Spectrum stoornis.

\subsection{Situatieschets Individuele begeleiding}
\label{sec:individu}
Één van de studenten die ik begeleid voor zijn afstuderen had een projectplan opgestuurd wat nogal onder de maat was. Erg algemene risico's niet echt specifieke doelen om te behalen en onderzoeksvraag waar niet echt mee gewerkt kon worden. Naar aanleiding van zijn projectplan en om even kennis te maken met de student, het afstudeerbedrijf en zijn begeleiders had ik een afspraak gemaakt met hem.

De afspraak begon helaas een beetje scheef omdat ik vergeten was de feedback die ik voor hem had verzameld een aantal dagen voor de afspraak naar hem te sturen. De student was een beetje verrast door mijn terugkoppeling op zijn document en de onderdelen die ik nog te onduidelijk vond. Het overviel hem even op dat moment, dat was totaal niet mijn bedoeling. 

Daarna heb ik geprobeerd door wat vragen te stellen duidelijker te krijgen wat de situatie was en wat de doelstelling van de opdracht was. Op een gegeven moment heb ik het gesprek afgerond en voorgesteld om nog even met de student individueel te zitten en samen te kijken naar de onderzoeksvraag en het werk wat er nog moet gebeuren.

De gesprekstechnieken die besproken zijn in de lessen en de oefening met de acteur waren mij allemaal even ontschoten en het kostte mij redelijk wat tijd om die weer op orde te krijgen. De taak als begeleider is naar mijn idee om de student zelf aan het denken te zetten. Op het moment dat de situatie dan geheel anders loopt, doordat de student de feedback pas voor het eerst hoorde toen ik bij hem was, dan is het toch een stuk lastiger om het ook daadwerkelijk bij de student te laten en niet zelf heel hard te gaan werken. 

Gezamenlijk hadden we afgesproken om elkaar nog telefonisch te spreken over het projectplan een week later. Tijdens dat gesprek was het een stuk makkelijker om de "goede" vragen te stellen en de student aan het werk te zetten. Hier blijkt voor mij duidelijk uit dat het belangrijk is om die gesprekstechnieken veelvuldiger te oefenen en te gebruiken zodat het een tweede natuur wordt, ook voor de onvoorziene situaties.

\subsection{Zelfreflectie}
In de gedeelten \hyperref[sec:groep]{Situatieschets Individuele begeleiding} en \hyperref[sec:individu]{Situatieschets Groepsbegeleiding} worden twee situaties aangedragen. Allereerst zou ik even de individuele situatie van de afstudeerder even willen uitlichten aan de hand van het estafetteloop-model \cite{methorstkaders}. Daarna behandel ik de groepssituatie aan de verschillende niveaus van reflectie van Korthagen \cite{korthagen2002niveaus}.

\subsubsection{Individuele situatie}
Het estafetteloop-model onderscheidt de fases van een opdracht in uitvoering bij een bedrijf met de metafoor van een estafetteloop. Waarin de opdracht het stokje voorstelt dat doorgegeven moet worden van bedrijf naar student weer terug naar het bedrijf. Als ik de opdracht van de student bekijk en de manier waarop hij in het begin bezig is geweest met het projectplan krijg ik het idee dat het doel en de eisen vanuit mij en vanuit de opdrachtgever wellicht niet duidelijk genoeg zijn geweest. 
Alhoewel de opdracht voldoet aan de kerneis van een praktijkopdracht binnen het idee van \textit{Slow Advice}: niet urgent, wel belangrijk \cite{methorstslow} is het niet duidelijk geworden voor de student welke onderzoeksvraag nou beantwoord moet worden. Als ik achteraf het gesprek bekijk in het licht van het estafetteloop-model zie ik dat ik een aantal dingen anders had moeten of kunnen doen.

Ten eerste had ik duidelijk moeten maken waar afstuderen aan moet voldoen. Zodat het doel van het document en zijn werk duidelijk is.
Ten tweede had ik de begeleiders aan het woord moeten laten door middel van vragen te stellen over het te worden opgeleverde product, bijvoorbeeld: \textit{Wat hebben jullie nodig om het een succes te maken?}. 

\subsubsection{Groepssituatie}
\emph{Fase 1: Handelen}

Het doel van de begeleiding van de projectgroep was enerzijds om de groep inzicht te geven in hoe er gewerkt wordt en anderzijds om het geven van feedback een plek te geven. Dit groepsoverleg was speciaal daarvoor gepland om dat bij de voorgaande groepsmomenten de feedback weinig diepgang had. Door iedereen wat op te laten schrijven en daarover door te vragen wilde ik voor iedereen een succes-ervaring creeëren in het geven van feedback.

\noindent \emph{Fase 2: Terugblikken op het handelen}
Een van de studenten probeerde onder de confrontatie uit te komen door alleen maar positieve dingen te noemen, wellicht voelde hij zich kwetsbaar of niet veilig, of dacht hij dat zijn ligging in de groep in het geding kwam door negatieve feedback te delen.
Door te intervenieren op het groepsgesprek heb ik getracht inzichtelijk maken dat deze persoon ook medeverantwoordelijk is voor het groepsproces. De condities voor het groepsproces kunnen zo gewaarborgd worden \cite{baert2009interventies}.

\noindent \emph{Fase 3: Bewust worden van essentiële aspecten}

Als ik terugkijk is het misschien niet duidelijk geweest wat de condities zijn voor een succesvol groepsproces. Als ik kijk naar de theorie is dat precies de rol die ik op mij had moeten nemen \cite{baert2009interventies}. Het ``contract'' was in die zin niet helder. Dat ga ik volgende keer anders doen door wat afspraken op te stellen van te voren.

\noindent \emph{Fase 4: Formuleren van alternatieven}

Natuurlijk ga ik niet een letterlijk contract opstellen, maar ik ga het wel duidelijker maken voor de leden voor de groep dat het gesprek er niet zozeer voor mij is maar voor hun eigen ontwikkeling. Bovendien denk ik dat het zou helpen om voor een aantal van de gesprekken een andere vorm te kiezen. Aan de hand van een metafoor of iets dergelijks (bijv. de groep is het schip, wat is het anker, wie is de stuurman etc.), ook wel de zijdeurinterventie genoemd \cite{remmerswaal2015zijdeurinterventies}.
